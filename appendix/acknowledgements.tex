% !Mode:: "TeX:UTF-8" 

\BiAppendixChapter{致\quad 谢}{Acknowledgements}

在此论文完成之际,谨向给予我无私帮助和关怀的人们致以最诚挚的谢意!

衷心感谢我的导师苏小红教授!本文是在苏老师的悉心指导下完成的,几年来苏老师对我的科研工作给予了大力支持,也在日常生活等等方面给予了无微不至的关怀。在攻读博士期间,论文的选题、开题、中期以及博士论文撰写的各个阶段,苏老师给予了细致且精心的指导,在此向她表达最衷心的感谢,她的言传身教将使我终生受益。苏老师渊博的知识、远见的学术洞察力、严谨的治学态度和执着的敬业精神是我受益匪浅。没有苏老师的悉心指导和热情鼓励,我的博士论文工作不可能如此顺利的完成。在此,谨向恩师致以由衷的敬意和衷心的感谢!

衷心感谢新加坡国立大学的KHOO Siau-Cheng教授!在新加坡国立大学访学期间,Prof. KHOO悉心地指导我进行学术研究,细致地与我讨论学术问题,认真的帮我修改学术论文。Prof. KHOO严谨的治学态度、卓越的学术能力将是我终生学习的方向!

衷心感谢所有对本文提出宝贵意见的专家们,尤其是责任专家、外审专家以及答辩专家!在所有专家的帮助、批评和指正下,使得本文更加完善。

衷心感谢实验室全体老师和同窗们的热情帮助和支持!感谢王甜甜老师、张彦航老师、赵玲玲老师!感谢实验室的师兄师姐师弟师妹们!

衷心感谢我的父母!感谢他们将我抚养成人,尽力的创造最好的条件和资源让我接受最好的教育,是他们对我一直以来的支持、关心、理解和厚望,鼓励和激励着我全身心的投入学习,让我有了最坚实的后盾,在遇到困难时从而能有继续前行的决心、勇气和动力。

博士只是人生不断学习的一个阶段,我将继续开启新的人生。以一句话勉励自身:“天行健,君子以自强不息;地势坤,君子以厚德载物!”
