% !Mode:: "TeX:UTF-8" 

\BiAppendixChapter{攻读\cxuewei 学位期间发表的论文及其他成果} {Papers
published in the period of PH.D. education}
\noindent\textbf{(一)已发表的(含录用)学术论文}
\begin{publist}

\item
\underline{Zhang Fanlong}, Khoo Siau-Cheng, Su Xiaohong{$^*$}. Predicting Consistent Clone Change[C]//Proceeding of the 27th International Symposium on Software Reliability Engineering (ISSRE), 2016: 353-364.(已发表,EI:20170803379101,CCF推荐B类会议,DOI:10.1109/ISSRE.2016.11,对应第4章,第一作者)
%Ottawa, Canada
\item
\underline{Zhang Fanlong}, Khoo Siau-cheng, Su Xiaohong{$^*$}. Machine-Learning Aided Analysis of Clone Evolution[J]. Chinese Journal of Electronics. (录用待发表, SCI收录, IF=0.513, 中科院分区4区, 对应第2章, 第一作者)

\item 
苏小红, \underline{张凡龙}{$^*$}. 面向管理的克隆代码研究综述 [J]. 计算机学报.(已录用, EI收录, 一级学报, 对应第1章, 第二作者, 导师为第一作者)

\item
\underline{Zhang Fanlong }, Su Xiaohong{$^*$},  Zhao Wen,  Ma Peijun. An Empirical Study of Code Clone Clustering Based on Clone Evolution[J]. Journal of Harbin Institute of Technology(New Series).2017, 24(2):10-18. (已发表, DOI:10.11916/j.issn.1005-9113.15316, 工大学报英文版, 对应第2章, 第一作者)

\item
\underline{张凡龙}, 苏小红{$^*$},  李智超,  马培军. 基于支持向量机的克隆代码有害性评价方法[J]. 智能计算机与应用, 2016, 6(4): 112-115. (已发表, 对应第3章, 第一作者)

\item
Su Xiaohong{$^*$}, \underline{Zhang Fanlong},  Xia Li, et al. Functionally Equivalent C Code Clone Refactoring by Combining Static Analysis with Dynamic Testing[C]//Proceeding of the International Conference on Soft Computing Techniques and Engineering Application. 2014: 247-256.(已发表, EI:20151600752325, DOI: 10.1007/978-81-322-1695-7\_28, 第二作者, 导师为第一作者)
\end{publist}

\noindent\textbf{(二)审稿中的学术论文}
\begin{publist}

\item
\underline{Zhang Fanlong}, Khoo Siau-Cheng, Su Xiaohong{$^*$}. Predicting Change Consistency in a Clone Group[J]. Journal of Systems and Software.(小修已修回, SCI收录, IF=2.444, CCF推荐B类期刊, 中科院SCI期刊分区3区, 对应第4章, 第一作者)

\item
\underline{Zhang Fanlong},  Khoo Siau-Cheng, Su Xiaohong{$^*$}. An Empirical Study on Clone Consistency-Requirement Prediction Based on Machine Learning[J]. Journal of Computer Science and Technology.(大修已修回, SCI收录, IF=0.956, CCF推荐B类期刊, 中科院SCI期刊分区4区, 对应第3章, 第一作者)

\item
\underline{Zhang Fanlong }, Su Xiaohong{$^*$}. Clone cross-project consistency prediction: an empirical study[J]. Software Quality Journal. (在审, SCI收录, IF=1.816, CCF推荐C类期刊, 中科院SCI期刊分区4区, 对应第5章, 第一作者)

\item
\underline{Zhang Fanlong}, Khoo Siau-cheng, Su Xiaohong{$^*$}. Improving Maintenance-Consistency Prediction During Code Clone Creation[J]. Chinese Journal of Electronics. (在审, SCI收录, IF=0.513, 中科院SCI期刊分区4区, 对应第3章, 第一作者)

%\item
%\textcolor{red}{\underline{Zhang Fanlong }, Su Xiaohong.  CCP: An Plug-in for Clone Consistency Prediction[C] (在投, 对应第5章, 第一作者)}
\end{publist}

\noindent\textbf{(三)与他人合作的学术论文}
\begin{publist}
\item
Yuan Yue, \underline{Zhang Fanlong},  Su Xiaohong{$^*$}. CloneAyz: An Approach for Clone Representation and Analysis[C]//Proceeding of the 3rd International Conference on Information Science and Control Engineering (ICISCE), 2016: 252-256.(已发表, EI:20165003106894, DOI:10.1109/ICISCE.2016.63, 第二作者)
\end{publist}

%%\noindent\textbf{(二)申请及已获得的专利(无专利时此项不必列出)}
%%\begin{publist}
%%\item XXX,XXX. 一种温热外敷药制备方案:中国,88105607.3[P]. 1989-07-26.
%%\end{publist}



\noindent\textbf{(四)参与的科研项目及获奖情况}
\begin{publist}
\item	
苏小红等. 无定型克隆代码的检测及重构方法. 国家自然科学基金项目. 课题编号:61173021.
\item 
苏小红等. 基于启发式选择变异和软件行为特征挖掘的软件错误定位方法. 国家自然科学基金项目. 课题编号:61672191.
\item
苏小红等. 数据挖掘和静态分析相结合的克隆代码缺陷检测及重构方法. 国家自然科学基金项目. 课题编号:61073052.
\item
苏小红等. 面向理解的软件错误定位方法. 国家自然科学基金项目. 课题编号:61202092.

\end{publist}
\vfill
\hangafter=1\hangindent=2em\noindent

\setlength{\parindent}{2em}
