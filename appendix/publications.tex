% !Mode:: "TeX:UTF-8" 

\BiAppendixChapter{攻读\cxuewei 学位期间发表的论文及其他成果} {Papers
published in the period of PH.D. education}
\noindent\textbf{(一)已发表(含接收)的学术论文}
\begin{publist}
%%\item	\underline{XXX},XXX. Static Oxidation Model of Al-Mg/C Dissipation Thermal Protection Materials[J]. Rare Metal Materials and Engineering, 2010, 39(Suppl. 1): 520-524.(SCI~收录,IDS号为~669JS,IF=0.16)
\item
\underline{Zhang Fanlong},Khoo Siau-Cheng,Su Xiaohong{$^*$}. Predicting Consistent Clone Change[C]. Software Reliability Engineering (ISSRE), 2016 IEEE 27th International Symposium on. IEEE, 2016: 353-364.(已发表,EI:20170803379101,CCF推荐B类会议,DOI:10.1109/ISSRE.2016.11,对应于第4章,第一作者)

\item
\underline{Zhang Fanlong},Khoo Siau-cheng, Su Xiaohong{$^*$}. Machine-Learning Aided Analysis of Clone Evolution[J]. Chinese Journal of Electronics. (录用待发表,SCI收录,IF=0.513,中科院分区四区,对应于第2章,第一作者)

\item 
苏小红,\underline{张凡龙}{$^*$}. 面向管理的克隆代码研究综述 [J]. 计算机学报.(已录用,EI~收录,一级学报,对应于第1章,第二作者,导师为第一作者)

\item
\underline{Zhang Fanlong },Su Xiaohong{$^*$}, Zhao Wen, Ma Peijun. An empirical study of code clone clustering based on clone evolution[J]. Journal of Harbin Institute of Technology(New Series). (已在线发表,DOI:10.11916/j.issn.1005-9113.15316,工大学报英文版,对应于第2章,第一作者)

\item
Yuan Yue,\underline{Zhang Fanlong}, Su Xiaohong{$^*$}. CloneAyz: An Approach for Clone Representation and Analysis[C]. Information Science and Control Engineering (ICISCE), 2016 3rd International Conference on. IEEE, 2016: 252-256.(已发表,EI:20165003106894,DOI:	10.1109/ICISCE.2016.63,对应于第5章,第二作者)

\item
\underline{张凡龙}, 苏小红{$^*$}, 李智超, 马培军. 基于支持向量机的克隆代码有害性评价方法[J]. 智能计算机与应用, 2016, 6(4): 112-115. (已发表,对应于第3章,第一作者)

\item
Su Xiaohong{$^*$},\underline{Zhang Fanlong}, Xia Li, et al. Functionally Equivalent C Code Clone Refactoring by Combining Static Analysis with Dynamic Testing[C]. Proceedings of International Conference on Soft Computing Techniques and Engineering Application. Springer India, 2014: 247-256.(已发表,EI:20151600752325,DOI: 10.1007/978-81-322-1695-7\_28,第二作者导师为第一作者)
\end{publist}

\noindent\textbf{(二)审稿中的学术论文}
\begin{publist}
\item
\underline{Zhang Fanlong}, Khoo Siau-Cheng,Su Xiaohong{$^*$}. Predicting Change Consistency in a Clone Group[J]. Journal of Systems and Software.(小修已修回,SCI收录,IF=2.444,CCF推荐B类期刊,中科院SCI期刊分区3区,对应于第4章,第一作者)

%%%\item
%%%\underline{Zhang Fanlong },Su Xiaohong{$^*$}. Clone cross-project consistency prediction: an empirical study[J]. Software Quality Journal. (在投,SCI收录,IF=1.816,CCF推荐C类期刊,中科院SCI期刊分区4区,对应于第5章,第一作者)

\item
\underline{Zhang Fanlong}, Khoo Siau-Cheng,Su Xiaohong{$^*$}. An empirical study on clone consistency-requirement prediction based on machine learning[J]. Journal of Computer Science and Technology.(在审,SCI收录,IF=0.956,CCF推荐B类期刊,中科院SCI期刊分区4区,对应于第3章,第一作者)

\item
\underline{Zhang Fanlong}, Khoo Siau-cheng, Su Xiaohong{$^*$}. Improving Maintenance-Consistency Prediction During Code Clone Creation[J]. Chinese Journal of Electronics. (在审,SCI收录,IF=0.513,中科院SCI期刊分区4区,对应于第3章,第一作者)

%\item
%\textcolor{red}{\underline{Zhang Fanlong },Su Xiaohong.  CCP: An Plug-in for Clone Consistency Prediction[C] (在投,对应于第5章,第一作者)}

\end{publist}

%%\noindent\textbf{(二)申请及已获得的专利(无专利时此项不必列出)}
%%\begin{publist}
%%\item XXX,XXX. 一种温热外敷药制备方案:中国,88105607.3[P]. 1989-07-26.
%%\end{publist}



\noindent\textbf{(三)参与的科研项目及获奖情况}
\begin{publist}
\item	
苏小红等. 无定型克隆代码的检测及重构方法. 国家自然科学基金项目. 课题编号:61173021.
\item 
苏小红等. 基于启发式选择变异和软件行为特征挖掘的软件错误定位方法. 国家自然科学基金项目. 课题编号:61672191.
\item
苏小红等. 数据挖掘和静态分析相结合的克隆代码缺陷检测及重构方法. 国家自然科学基金项目. 课题编号:61073052.
\item
苏小红等. 面向理解的软件错误定位方法. 国家自然科学基金项目. 课题编号:61202092.

\end{publist}
\vfill
\hangafter=1\hangindent=2em\noindent

\setlength{\parindent}{2em}
