% !Mode:: "TeX:UTF-8" 

\BiAppendixChapter{攻读\cxuewei 学位期间发表的论文及其他成果} {Papers
published in the period of PH.D. education}
\noindent\textbf{(一)已发表的(含录用)学术论文}
\begin{publist}

\item
\underline{Zhang Fanlong}, Khoo Siau-Cheng, Su Xiaohong{$^*$}. Predicting Change Consistency in a Clone Group[J]. Journal of Systems and Software. 134(2017), 105-119.
(已发表, DOI:10.1016/j.jss.2017.08.045, SCI收录,  5-Year IF=2.619, CCF推荐B类期刊, 中科院SCI期刊分区3区, 对应第4章, 第一作者)
%2016年IF=2.444, 

\item
\underline{Zhang Fanlong}, Khoo Siau-Cheng, Su Xiaohong{$^*$}. Predicting Consistent Clone Change[C]. Proceeding of the 27th International Symposium on Software Reliability Engineering (ISSRE), Ottawa, Canada, 2016: 353-364.
(已发表, DOI:10.1109/ISSRE.2016.11, EI:20170803379101, CCF推荐B类会议, 对应第4章, 第一作者)

\item
\underline{Zhang Fanlong}, Khoo Siau-cheng, Su Xiaohong{$^*$}. Machine-Learning Aided Analysis of Clone Evolution[J/OL]. Chinese Journal of Electronics: 1-7(2017-09-28). http://kns.cnki.net/kcms/detail/10.1284.TN.20170928.1353.002.html.
(已发表, DOI:10.1049/cje.2017.08.012, SCI收录, 2016年IF=0.513, 中科院分区4区, 对应第2章, 第一作者)

%%%\item
%%%\underline{Zhang Fanlong}, Khoo Siau-cheng, Su Xiaohong{$^*$}. Machine-Learning Aided Analysis of Clone Evolution[J]. Chinese Journal of Electronics.  26(2017).
%%%(已发表, DOI:10.1049/cje.2017.08.012, SCI收录, 2016年IF=0.513, 中科院分区4区, 对应第2章, 第一作者)

\item
苏小红, \underline{张凡龙}{$^*$}. 面向管理的克隆代码研究综述[J/OL]. 计算机学报2017(2017-08-24). On Publishing: No.120, Vol.40. http://kns.cnki.net/kcms/detail/11.1826.TP.20170728.1305.046.html.
(已发表, EI收录, 一级学报, 对应第1章, 第二作者, 导师为第一作者)

%%%\item
%%%苏小红, \underline{张凡龙}{$^*$}. 面向管理的克隆代码研究综述[J/OL]. 计算机学报2017(2017-07-28) [2017-08-24]. http://kns.cnki.net/kcms/detail/11.1826.TP.20170728.1305.046.html.
%%%(网络优先发表, EI收录, 一级学报, 对应第1章, 第二作者, 导师为第一作者)

\item
\underline{Zhang Fanlong }, Su Xiaohong{$^*$},  Zhao Wen,  Ma Peijun. An Empirical Study of Code Clone Clustering Based on Clone Evolution[J]. Journal of Harbin Institute of Technology(New Series).2017, 24(2):10-18.
(已发表, DOI:10.11916/j.issn.1005-9113.15316, 哈工大学报英文版, 对应第2章, 第一作者)

\item
\underline{张凡龙}, 苏小红{$^*$},  李智超,  马培军. 基于支持向量机的克隆代码有害性评价方法[J]. 智能计算机与应用, 2016, 6(4): 112-115. 
(已发表, 对应第3章, 第一作者)

\item
Su Xiaohong{$^*$}, \underline{Zhang Fanlong},  Xia Li, et al. Functionally Equivalent C Code Clone Refactoring by Combining Static Analysis with Dynamic Testing[C]. Proceeding of the International Conference on Soft Computing Techniques and Engineering Application. 2014: 247-256.
(已发表, DOI: 10.1007/978-81-322-1695-7\_28, EI:20151600752325, 第二作者, 导师为第一作者)
\end{publist}

\noindent\textbf{(二)审稿中的学术论文}
\begin{publist}

\item
\underline{Zhang Fanlong},  Khoo Siau-Cheng, Su Xiaohong{$^*$}. An Empirical Study on Clone Consistency-Requirement Prediction Based on Machine Learning[J]. Journal of Computer Science and Technology.
(大修, SCI收录, 2016年IF=0.956, CCF推荐B类期刊, 中科院SCI期刊分区4区, 对应第5章, 第一作者)

\item
\underline{Zhang Fanlong}, Khoo Siau-cheng, Su Xiaohong{$^*$}. Improving Maintenance-Consistency Prediction During Code Clone Creation[J]. Software Quality Journal. 
(在审, SCI收录, 2916年IF=1.816, CCF推荐C类期刊, 中科院SCI期刊分区4区, 对应第3章, 第一作者))

\item
\underline{Zhang Fanlong }, Su Xiaohong{$^*$}. Clone Consistency prediction for Cross-Project: An Empirical Study[J]. Journal of Systems and Software.
(在审, SCI收录, 5-Year IF=2.619, 2016年IF=2.444, CCF推荐B类期刊, 中科院SCI期刊分区3区, 对应第5章, 第一作者)
%Cluster Computing. (在审, SCI收录, 2016年IF=2.040, 中科院SCI期刊分区3区, 对应第5章, 第一作者)

%\item 放弃
%\underline{Zhang Fanlong }, Su Xiaohong.  CCP: An Plug-in for Clone Consistency Prediction[C] 
%(在投, 对应第5章, 第一作者)

%\item 放弃
%\underline{张凡龙 }, 何蔷, 苏小红{$^*$}. 克隆代码可视化方法研究[J].哈尔滨工业大学学报.
% (在投, EI收录, 第一作者)

\end{publist}

\noindent\textbf{(三)与他人合作的学术论文}
\begin{publist}
\item
Yuan Yue, \underline{Zhang Fanlong},  Su Xiaohong{$^*$}. CloneAyz: An Approach for Clone Representation and Analysis[C]. Proceeding of the 3rd International Conference on Information Science and Control Engineering (ICISCE), 2016: 252-256.(已发表, EI:20165003106894, DOI:10.1109/ICISCE.2016.63, 第二作者)
\end{publist}

%%\noindent\textbf{(二)申请及已获得的专利(无专利时此项不必列出)}
%%\begin{publist}
%%\item XXX,XXX. 一种温热外敷药制备方案:中国,88105607.3[P]. 1989-07-26.
%%\end{publist}
\noindent\textbf{(四)参与的科研项目}
\begin{publist}

\item
苏小红等. 建筑全性能联合仿真平台内核开发. “十三五”国家重点研发计划课题. 课题编号:2017YFC0702204.
\item	
苏小红等. 基于启发式选择变异和软件行为特征挖掘的软件错误定位方法. 国家自然科学基金项目. 课题编号:61672191.
\item	
苏小红等. 无定型克隆代码的检测及重构方法. 国家自然科学基金项目. 课题编号:61173021.
\item
苏小红等. 数据挖掘和静态分析相结合的克隆代码缺陷检测及重构方法. 国家自然科学基金项目. 课题编号:61073052.
\item
苏小红等. 面向理解的软件错误定位方法. 国家自然科学基金项目. 课题编号:61202092.

\item
苏小红等. 基于程序转换和语义分析的编程自动评分方法研究. 国家自然科学基金项目. 课题编号:60673035.

\end{publist}
\vfill
\hangafter=1\hangindent=2em\noindent

\setlength{\parindent}{2em}
