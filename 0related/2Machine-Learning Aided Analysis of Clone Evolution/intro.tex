%\begin{center}{\large\bf I.\ Introduction}\end{center}
\begin{center}{\large\bf 1.\ Introduction}\end{center}

Code clones are collection of code fragments which are similar to one another according to some definition of ``similarity''[1]. They may take up 7-23\% of the software, even 59\% in some software[2]. %Clones are often the result of copy-and-paste activities, and there are several other factors for their existence, such as adoption of reuse approach, programming approach, and maintenance benefits. 
%Researchers propose several methods to detect clones, such as NiCad. which can be divided into Text-based, Token-based, AST-based, PDG-based and others according to their technology. With appropriate clone detection tools (such as NiCad), we can detect clones within a software.  
As clones always evolve with the softwares, Kim et al. propose clone genealogies to describe this evolution process, and provide several clone patterns to categorize different clone change during their evolution[3]. This has led to a new research direction focusing on clone evolution and clone patterns. Currently, there are tools available for detecting clones and building clone genealogies. Their availability has led to numerous work on clone analyses, such as studies on clone stability and clone patterns [5,6]. %This can guide developers to understand the clones. 

At the same time, there has been contraversial debates over the usefulness of clones in software. Some researchers opine that the presence of clones is harmful, as they can incur additional maintenance effort [4], and even cause software defects. For instance, clones that change inconsistently during software evolution are likely to lead to defects [5]. Consequently, developers should take measures to avoid clone creation, remove clones through refactoring. Other researchers take a positive outlook towards the existence of clones. They view code clones as mainly a by-product of reuse, and most of the clones created turn out to be more stable than non-clones, and thus can save much development time [6]. 

%A good understanding of clones and their evolution can provide strong assistance to software maintenance.  
Regardless of different opinions, clones can naturally be found in software, as ``copy-and-paste" way of producing code (and thus creating clones) remains efficient in software development. It is therefore important to understand clone characteristics and how they impact development and maintenance in software evolution. Such investigation result can have substantive help on clone anlaysis and management.    Specifically, 
We are interested in deriving relationships; we name these relationships {\em clone evolutionary characteristic}. 
They can offer fresh insight from the evolutionary aspect of clones in a piece of software. Throughout the paper, we use the terms ``clone evolutionary characteristics'', ``clone characteristics'' and ``characteristics'' interchangeably.  

%We consider three different perspectives from three different clone entities: clone fragment, clone group and clone genealogy. Clone fragment offers an individual perspective about clones. It tells us how individual clones are changed during evolution. Clone group offers a regional perspective about clones. %It shows us how a group of similar clones is being operated upon during evolution. Clone genealogy offers a global perspective about clones. %It tells us how clone groups are evolved from beginning till the end of our investigations.

In this paper, we propose an approach to explore and analyze clone characteristics through a machine learning method. We build clone genealogies through mapping clones between neighboring versions. After that, we extract several metrics at three levels to represent three different clone entities: {\em Clone Fragment, Clone Group and Clone Genealogy}. These metrics can cover many interesting and relevant information about clone entities. Lastly, we use  X-means clustering implemented by WEKA (short for ``Waikato Environment for Knowledge Analysis''[25]) to cluster all the clones.
We conduct an empirical study on two open-source software packages. 

Our study 
shows that clones are in general very stable during their evolution, and clones usually do not undergo changes at the infancy stage of evolution. Developers should therefore pay more attention to (more matured) clones that have existed in a genealogy after several evolutions (aka., in longer life clone genealogy). We also suggest that developers should consider the possibility of making consistent changes across the entire clone group when one of the constituent clone fragments has undergone change.  The contributions of this paper are as follows:
\begin{enumerate}
\setlength{\itemsep}{0pt}
\setlength{\parsep}{0pt}
\setlength{\parskip}{0pt}
\item  We propose a framework to analyze the clones to explore clone evolutionary characteristics, which are meaningful to clone analysis and maintenance.
\item We view clones as one kind of data from three different perspectives: clone fragment, clone group and clone genealogy, and extract some metrics to present the clones respectively.
\item  We conducted an empirical study on two open source software, and obtain interesting findings which can give developers some suggestions to better understanding and maintaining clones in particular, and software in general.
\end{enumerate}

Our paper is organizing as follows. Section 2 is the related work. %We give a brief introduction of clone research. %Section 3 is our motivation. We explain why we want to do this work. We explain the clone evolution characteristic of clones. 
Section 3 is our method which explain how to analyses the clones. %Details can be seen in the following parts in this paper. 
Section 4 is our case study. %We give the experiment systems, and the steps we take in the experiment. 
Section 5 is our analysis and results. %We analyze the results about clone characteristic. 
Section 6 is discussion. Section 7 is our conclusion. 

%\begin{center}{\large\bf III.\ Motivation}\end{center}

%With appropriate clone detection tools (such as NiCad), we can detect clones within a software. Clones are enveloping with software. Clone genealogies can describe clone evolution from successive versions of software, and clone patterns can describe clone change during their evolution. We believe that there should be some relationship in clones and their evolution. However, how to analyze these clones and their evolution to find the relationships, which can help developers understand clones? In our earlier work, we clustered clone fragments to find relationships between clone lifes and clone patterns, and derive that the short-life clone fragment may be more frequently changed than the long-life ones. We believe more can be derived/inferred from a good collection of information extracted from clone groups and their genealogy. Specifically, we are interested in deriving relationships between clones and their evolution; we name these relationships {\em clone evolutionary characteristic}.

%Clone evolutionary characteristic is the clone relationships between the clones and their evolution, which can help developers to understand and maintain the clones. Clone evolutionary characteristics adds new values to existing clone analysis and maintenance tasks, as it offers fresh insight from the evolutionary aspect of clones in a piece of software. Throughout the paper, we use the terms ``clone evolutionary characteristics'', ``clone characteristics'' and ``characteristics'' interchangeably. 

%In order to explore more clone characteristics from clones and evolution, we consider three different perspectives from three different clone entities: clone fragment, clone group and clone genealogy. A clone fragment offers an individual perspective about clones. It tells us how individual clones are changed during evolution. Clone groups offer a regional perspective about clones. It shows us how a group of similar clones is being operated upon during evolution. Lastly, a clone genealogy offers a global perspective about clones. It tells us how clone groups are evolved from beginning till the end of our investigations. 

%In this work, we adopt a machine-learning algorithm named X-means to the discovery of clone evolution characteristics. We shall describe this approach in detail in the following section.
