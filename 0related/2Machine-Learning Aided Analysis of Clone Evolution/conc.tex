\begin{center}{\large\bf 7.\ Conclusion}\end{center}

In this paper, we propose an approach to analyze clone evolutionary characteristics through machine learning method from clones and their evolution. Specifically, we have demonstrated the usefulness of clustering (via X-means algorithm) the clone entities by taking into consideration the life times of clone entities. %The contributions of our paper include: (1) we propose a framework to analyze the clone characteristics for multi-version softwares. (2) We extract a class of relevant clone metrics to characterize clone fragment, clone group and clone genealogy. We provide three different perspectives about clone evolution, so that conclusions can be drawn from all aspects of clone evolution, from individual clone perspective to global-based genealogy perspective. (3) We conduct a case study on two softwares: {\em ArgoUML} and {\em jEdit} to explore the clone evolutionary characteristics. 
Our case study shows that clones are in general very stable during their evolution, and clones usually do not undergo changes at the infancy stage of evolution. Developers should therefore pay more attention to (more matured) clones that have existed in a genealogy after several evolutions (aka., in longer life clone genealogy). We also suggest that developers should consider the possibility of making consistent changes across the entire clone group when one of the constituent clone fragments has undergo change. 

We believe that these characteristics can help developers to understand clones better, and can also provide some guidance to maintain and manage clones in software development. In future, we intend to improve our analysis on clone group and clone genecology by incorporating more metrics. The conclusion drawn about the relationships on clone group and genealogy can be viewed as the features of clone evolution. These features may be helpful for clone management.

