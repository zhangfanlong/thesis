% !Mode:: "TeX:UTF-8" 

\newcommand{\chinesethesistitle}{基于机器学习的克隆代码分析和维护方法研究} %授权书用,无需断行
\newcommand{\englishthesistitle}{\uppercase{RESEARCH ON CODE CLONE ANALYSIS AND MAINTEANCE BASED ON MACHINE LEARNING}} %\uppercase作用:将英文标题字母全部大写;
\newcommand{\chinesethesistime}{2017 年~7 月}  %封面底部的日期中文形式
\newcommand{\englishthesistime}{July, 2017}    %封面底部的日期英文形式

\ctitle{基于机器学习的克隆代码分析和维护方法研究}  %封面用论文标题,自己可手动断行
\cdegree{\cxueke\cxuewei}
\csubject{计算机科学与技术}                 %(~按二级学科填写~)
\caffil{计算机科学与技术学院} %(在校生填所在系名称,同等学力人员填工作单位)
\cauthor{张凡龙}
\csupervisor{苏小红教授} %导师名字
%\cassosupervisor{副导名}%若没有,请屏蔽掉此句。
%\ccosupervisor{联导名}%若没有,请屏蔽掉此句。


\cdate{\chinesethesistime}

\etitle{\englishthesistitle}
\edegree{\exuewei \ of \exueke}
\esubject{\emultiline[t]{Computer Science\\ and Technology}}  %英文二级学科名
\eaffil{School of Computer Science and Technology}
\eauthor{Zhang Fanlong}                   %作者姓名 (英文)
\esupervisor{Prof. Su Xiaohong}       % 导师姓名 (英文)
%\eassosupervisor{Prof. Assosuper}%若没有,请屏蔽掉此句。
%\ecosupervisor{Prof. Cosuper}%若没有,请屏蔽掉此句。
\edate{\englishthesistime}

\natclassifiedindex{TP315.2}  %国内图书分类号
\internatclassifiedindex{62-5}  %国际图书分类号
\statesecrets{公开} %秘密

\iffalse
\BiAppendixChapter{摘~~~~要}{}
\fi
\cabstract{
克隆代码是存在于软件中的彼此相似的代码片段。研究表明软件中存在大量的克隆代码,其存在会影响软件质量、可理解性和可维护性。在软件演化的过程中,克隆代码也会着软件进行演化,这一过程可以使用克隆家系来描述。在克隆代码的演化过程中,克隆代码往往表现出一些特征,本文称之为克隆代码演化特征。其中,引发研究人员强烈关注的是克隆代码在演化过程中的变化,即一致性变化和不一致变化。克隆的变化会导致与之相关的克隆缺陷以及额外的维护代价,从而降低软件质量。因此,本文在软件演化的基础上研究克隆代码的分析和维护方法,旨在帮助软件开发和维护人员理解和维护克隆代码,帮助提高软件质量和可维护性。

具体地,本文研究基于机器学习的克隆代码分析和维护方法,通过提取克隆代码相关的特征用于表示克隆代码,并结合相关的机器学习方法解决克隆代码的分析和维护问题。
研究了基于聚类的克隆代码演化特征提取方法。克隆代码在其演化过程中会表现出来一些特征,本文称之为克隆演化特征,可以帮助程序开发和维护人员理解和维护克隆代码。本文在提取相应度量值得基础上使用聚类方法从克隆片段、克隆组和克隆家系三个维度进行实证研究。研究发现大部分的克隆代码在演化过程中是稳定的,但也有相当数量的克隆代码发生变化,尤其是发生一致性变化的克隆代码会比不一致的克隆代码多一些。
然而,克隆代码的一致性变化往往会导致额外的维护代价,而遗忘一致性变化通常又会导致克隆缺陷的产生。克隆代码的一致性变化需要引起开发人员和维护人员的关注。本文将克隆代码的一致性变化称为克隆一致性维护需求。鉴于此,为了帮助开发人员和维护人员避免克隆一致性变化及其相关缺陷,本文使用贝叶斯网络对克隆代码的一致性变化进行预测,分别在克隆代码产生和发生变化时预测克隆代码的一致性维护需求。对于新产生的克隆代码,在复制粘贴时通过提取被复制和被粘贴的克隆代码的两组属性值预测克隆代码的一致性维护需求,可以帮助避免克隆代码。对于系统中已经存在的克隆代码,在克隆代码发生变化时通过提取三组不同的属性预测其一致性变化需求,可帮助提示程序员注意克隆代码的变化。研究表明本文的预测方法可以有效地预测克隆代码的一致性维护需求,可以帮助避免软件缺陷和额外的维护代价。
进一步地,为帮助程序人员选择合理的一致性需求预测模型,本文还进行了克隆一致性维护需求的实证研究。将克隆代码的一致性维护预测扩展到其它的机器学习方法中, 并给出了相应的建议指导开发人员进行选择。同时,还对比了复制粘贴时和克隆变化时的预测,也给出了相关的建议。

本文基于机器学习的克隆代码分析和维护方法,不仅可以帮助软件开发人员分析和理解克隆代码,还可以帮助程序开发人员维护克隆代码。本文的研究有较强的实际意义,可以帮助避免克隆代码的相关缺陷,降低克隆代码的维护代价。同时,本文方法可以嵌入到软件开发环境中实现边开发、边分析、边维护克隆代码,帮助提高软件的质量和可维护性。%本文通过克隆演化特征分析获取了克隆代码的演化特征,基于克隆代码的一致性变化问题,提出了不同时刻的一致性维护需求预测方法,并在不同的实验系统上进行了实证研究,取得了较好的研究成果。
}

\ckeywords{克隆代码;克隆分析;克隆维护;机器学习;一致性变化;软件演化}

\eabstract{
Code Clone are the simlar code fragments which are introduce by copy-and-paste operstions. 
The absenence of clones in software may impact software quliaty and maintanence.
Clones always evolving with software, there are some features which called clone evolutionary charactics. 
Clone .... can help developer to understand and maintain the clones, which 
}

\ekeywords{keyword 1, keyword 2, keyword 3, ……, keyword 6 (no punctuation at the end) 英文摘要与中文摘要的内容应一致,在语法、用词上应准确无误。}

\makecover
\clearpage 
