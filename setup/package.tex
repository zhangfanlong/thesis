% !Mode:: "TeX:UTF-8"
\usepackage{xeCJK}
\usepackage{tabularx}
\usepackage{graphicx}
\usepackage{multirow}
\usepackage{rotating}
%\usepackage[figuresright]{rotating}
\usepackage[a4paper,text={150true mm,224true mm},top=35.5true mm,left=30true mm,head=5true mm,headsep=2.5true mm,foot=8.5true mm]{geometry}
\usepackage{titlesec}               % 控制标题的宏包
\usepackage{titletoc}                   % 控制目录的宏包
\usepackage{fancyhdr}                   % fancyhdr宏包 页眉和页脚的相关定义
\usepackage{color}          % 支持彩色
\usepackage{amsmath}        % AMSLaTeX宏包 用来排出更加漂亮的公式
\usepackage{amssymb}
\usepackage[below]{placeins}%允许上一个section的浮动图形出现在下一个section的开始部分,还提供\FloatBarrier命令,使所有未处理的浮动图形立即被处理
\usepackage{flafter}       % 使得所有浮动体不能被放置在其浮动环境之前,以免浮动体在引述它的文本之前出现.
\usepackage{multirow}       %使用Multirow宏包,使得表格可以合并多个row格
\usepackage{booktabs}       % 表格,横的粗线;\specialrule{1pt}{0pt}{0pt}
\usepackage{longtable}      %支持跨页的表格。
\usepackage[hang]{subfigure}%支持子图 %centerlast 设置最后一行是否居中
\usepackage[subfigure]{ccaption} %支持双语标题
\usepackage[sort&compress,numbers]{natbib}% 支持引用缩写的宏包
\usepackage{enumitem}       %使用enumitem宏包,改变列表项的格式
\usepackage{calc}           %长度可以用+ - * / 进行计算
%\usepackage{txfonts}
\usepackage{fontspec}
\usepackage[amsmath,thmmarks,hyperref]{ntheorem}% 定理类环境宏包,其中 amsmath 选项用来兼容 AMS LaTeX 的宏包
\usepackage{etoolbox}
\AtBeginDocument{%
  \apptocmd\thebibliography{\interlinepenalty=-5000 }{}{}%
}

% 生成有书签的pdf及其开关, 该宏包应放在所有宏包的最后, 宏包之间有冲突
\def\atemp{dvipdfmx}\ifx\atemp\usewhat
\usepackage[dvipdfmx,unicode,           %dvi-->pdf 生成书签
            bookmarksnumbered=true,
            bookmarksopen=true,
            colorlinks=false,
            pdfborder={0 0 1},
            citecolor=blue,
            linkcolor=red,
            anchorcolor=green,
            urlcolor=blue,
            breaklinks=true
            ]{hyperref}
\fi

\def\atempxetex{xelatex}\ifx\atempxetex\usewhat %\def\atempxetex{xelatex} main.tex中已定义;
\usepackage[xetex,
            bookmarksnumbered=true,
            bookmarksopen=true,
            colorlinks=false,
            pdfborder={0 0 1},
            citecolor=blue,
            linkcolor=red,
            anchorcolor=green,
            urlcolor=blue,
            breaklinks=true,
            naturalnames  %与algorithm2e宏包协调
            ]{hyperref}

\defaultfontfeatures{Mapping=tex-text}
\xeCJKsetemboldenfactor{1}%只对随后定义的CJK字体有效
\setCJKfamilyfont{hei}{SimHei}
\xeCJKsetemboldenfactor{4}
\setCJKfamilyfont{song}{SimSun}
\xeCJKsetemboldenfactor{1}
\setCJKfamilyfont{fs}{FangSong}
\setCJKfamilyfont{kai}{KaiTi}
\setCJKfamilyfont{li}{LiSu}
\setCJKfamilyfont{xw}{STXinwei}
\setCJKmainfont{SimSun}
\setCJKsansfont{SimHei}
\setmainfont{Times New Roman}
\setsansfont{Arial}
\newcommand{\hei}{\CJKfamily{hei}}% 黑体   (Windows自带simhei.ttf)
\newcommand{\song}{\CJKfamily{song}}    % 宋体   (Windows自带simsun.ttf)
\newcommand{\fs}{\CJKfamily{fs}}        % 仿宋体 (Windows自带simfs.ttf)
\newcommand{\kaishu}{\CJKfamily{kai}}      % 楷体   (Windows自带simkai.ttf)
\newcommand{\li}{\CJKfamily{li}}        % 隶书   (Windows自带simli.ttf)
\newcommand{\xw}{\CJKfamily{xw}}        % 隶书   (Windows自带simli.ttf)
\newfontfamily\arial{Arial}
\newfontfamily\timesnewroman{Times New Roman}
\fi

\usepackage[plainruled,linesnumbered,algochapter]{algorithm2e}  % 算法的宏包,注意宏包兼容性,先后顺序为float、hyperref、algorithm(2e),否则无法生成算法列表
\usepackage{xltxtra}
\usepackage{listings}
\lstset{
%basicstyle=\small\ttfamily,
columns=flexible,
breaklines=true
}
\newcommand{\emultiline}[2][c]{\renewcommand{\arraystretch}{1}\begin{tabular}[#1]{@{}l@{}}#2\end{tabular} \renewcommand{\arraystretch}{1.3} }
%\newcommand{\citeayu}[1]{\citeauthor{#1}~(\citeyear{#1})\citeup{#1}}

