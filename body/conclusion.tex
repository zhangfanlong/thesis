% !Mode:: "TeX:UTF-8" 

\BiAppendixChapter{结\quad 论}{Conclusions}

%学位论文的结论作为论文正文的最后一章单独排写,但不加章标题序号。

%结论应是作者在学位论文研究过程中所取得的创新性成果的概要总结,不能与摘要混为一谈。博士学位论文结论应包括论文的主要结果、创新点、展望三部分,在结论中应概括论文的核心观点,明确、客观地指出本研究内容的创新性成果(含新见解、新观点、方法创新、技术创新、理论创新),并指出今后进一步在本研究方向进行研究工作的展望与设想。对所取得的创新性成果应注意从定性和定量两方面给出科学、准确的评价,分(1)、(2)、(3)…条列出,宜用“提出了”、“建立了”等词叙述。

软件中克隆代码随着系统演化,克隆代码的存在以及在演化过程中的变化,会导致软件难以理解和维护,已经成为了影响软件质量的一个重要因素。现有的研究方法中既不能客观全面的理解克隆代码,也不能有效地解决克隆代码一致性变化问题。针对以上问题,本文提出了基于软件演化和机器学习的克隆代码分析与一致性维护方法,重点研究了克隆代码演化特征分析、克隆代码创建时和变化时的一致性需求预测,并结合软件开发过程中对克隆代码一致性预测进行实证研究,取得了如下创新性成果:

(1) 针对现有的克隆分析方法不能全面客观地理解克隆代码及其演化过程,提出了一个基于聚类的克隆代码演化特征分析方法。方法通过检测软件版本中的克隆代码,并构建系统克隆家系描述克隆代码的演化过程。为表示克隆代码及其演化情况,从克隆片段、克隆组和克隆家系三个不同的角度提取相应的属性值。在此基础上使用X-means聚类分析和挖掘克隆代码的演化特征。实验表明软件中存在的大部分克隆代码在其演化过程中是较为稳定的,并不会发生频繁的变化。同时,也存在一定规模的发生变化的克隆代码,其中发生一致性变化的克隆要多于不一致变化的克隆代码。

(2) 针对复制粘贴创建的克隆代码在其演化过程可能会引发一致性变化,从而导致额外的维护代价问题,提出了一个基于贝叶斯网络的克隆代码创建时一致性维护需求预测方法。提供了一个克隆代码创建时的一致性维护需求定义,将问题转化为一个可以使用机器学习解决的分类问题。通过构建系统克隆家系并识别克隆演化模式,从系统中收集克隆创建实例用于训练贝叶斯网络模型,并提取代码属性表示被复制的克隆代码、提取上下文属性表示被粘贴的克隆代码。在四个实验系统的实验表明,该方法可以以较高的准确率和召回率高效地预测克隆代码创建时的一致性维护需求。

(3) 针对演化过程中的克隆代码的变化会导致其所在克隆组在演化过程中的一致性变化,从而可能引发的一致性缺陷问题,提出了一个基于贝叶斯网络的克隆代码变化时一致性维护需求预测方法。提供了一个克隆代码变化时的一致性维护需求定义,将问题装化为可以应用机器学习方法解决的分类问题。通过构建系统的克隆家系和识别其演化模式,从系统中收集克隆变化实例用于训练贝叶斯网络模型。从克隆组的角度提取了代码属性、上下文属性,并同时从克隆演化的角度提取了一组演化属性共同表示克隆变化实例。在四个实验系统的实验表明,该方法可以以合理的准确率和召回率有效地预测克隆代码变化使得一致性维护需求。

(4) 针对如何结合软件开发过程并帮助程序开发人员选择合适的机器学习方法对克隆代码进行一致性维护的问题,对克隆代码的一致性维护需求进行了一个实证研究。从实际应用的角度提出一个克隆一致性预测的研究问题,并统一克隆创建和变化时的一致性维护需求为克隆代码一致性维护需求定义。结合软件开发过程,基于eclipse设计并实现了一个克隆代码一致性维护需求预测插件,可以在实际开发中帮助程序开发人员预测克隆代码的一致性维护需求。并使用五种不同的机器学习方法预测克隆代码的一致性维护需求,从而给出在实际开发过程中如何预测克隆代码一致性维护需求。该方法可以在软件开发过程中帮助程序开发人员边开发、边预测克隆代码的一致性维护需求。

在本文工作的基础上,还有以下工作有待于进一步研究:

(1) 在克隆变化时一致性维护需求预测中,仍然有一定的改进空间。在未来研究中,拟改进现有的属性值用于表示克隆代码变化实例,从而以更高的准确率和召回率更加高效的对克隆代码的变化进行一致性维护需求预测。

(2)本文方法在对跨项目的克隆一致性维护需求预测时,现有方法的预测能力远远达不到令人满意的效果。在未来的研究中,拟通过添加与项目自身相关的属性用于跨项目的克隆代码一致性维护需求预测中,从而帮助缺乏训练数据的新系统进行一致性维护需求预测。

(3)在软件开发过程中,本文方法仅可以警告程序开发人员克隆代码的一致性维护需求,并不能切实的对克隆代码进行一致性维护。在未来的研究中,拟结合程序分析技术在需要时对克隆代码自动地进行一致性维护,保证克隆代码的一致性。
