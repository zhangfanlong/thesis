% !Mode:: "TeX:UTF-8" 

\BiAppendixChapter{结\quad 论}{Conclusions}

%学位论文的结论作为论文正文的最后一章单独排写,但不加章标题序号。

%结论应是作者在学位论文研究过程中所取得的创新性成果的概要总结,不能与摘要混为一谈。博士学位论文结论应包括论文的主要结果、创新点、展望三部分,在结论中应概括论文的核心观点,明确、客观地指出本研究内容的创新性成果(含新见解、新观点、方法创新、技术创新、理论创新),并指出今后进一步在本研究方向进行研究工作的展望与设想。对所取得的创新性成果应注意从定性和定量两方面给出科学、准确的评价,分(1)、(2)、(3)…条列出,宜用“提出了”、“建立了”等词叙述。


将克隆分析和克隆维护有机的结合为一个统一的整体,在软件开发过程中进行克隆代码的实时维护,通过监听克隆代码的变化指导软件的开发,在克隆分析、评价和克隆可视化的基础上,实现对克隆代码的分类维护,利用克隆代码的分析结果指导克隆代码的维护过程,从而实现了边开发边分析以及边开发边维护,为克隆代码的分析和维护提供了一种新的思路。
(1)将克隆代码视为一种数据,使用机器学习方法对克隆代码分析,试图揭示克隆代码及其演化过程所隐含的信息,并在此基础上评价克隆代码的有害性和有益性。
(2)将克隆可视化结合到克隆代码分析中,使用可视化方法可视化克隆代码信息帮助理解和分析克隆代码;同时,将克隆可视化结合到克隆代码维护过程中,使用可视化方法帮助维护克隆代码。
(3)将克隆代码分析和维护过程结合到软件开发过程中,实现边开发边分析,边开发边维护的克隆代码分析和维护:能够结合软件开发过程和克隆代码分析,实现在软件开发过程中实现对克隆代码的维护,在软件维护阶段实现对克隆代码的缺陷维护与复用维护,在软件编码阶段实现克隆代码产生与变化的实时维护。
6.2 研究内容与研究方案上的创新
(1)在提取克隆度量的基础上,使用机器学习方法对克隆代码进行聚类分析,揭示了克隆代码的演化特征,帮助更好的理解克隆代码。
(2)在提取克隆演化特征的基础上,使用机器学习方法对克隆代码进行评价分析,在海量的克隆代码中识别重要的克隆代码,并分析评价其有害性与有益性。
(3)在克隆有害性和有益性分析的基础上,将克隆维护与软件开发过程相结合,在软件开发过程中实现对克隆代码的缺陷维护和复用维护,辅助软件维护人员提高软件质量,降低软件维护的成本。
