% !Mode:: "TeX:UTF-8" 

\BiAppendixChapter{结\quad 论}{Conclusions}

克隆代码在随着软件系统进行演化的过程中,克隆代码的演化过程以及可能发生的一致性变化问题,导致软件越来越难以理解和维护,已经成为了影响软件质量的一个重要因素。现有的研究方法既不能客观全面的理解克隆代码,也不能有效地解决克隆代码一致性变化问题。针对上述问题,本文提出了基于软件演化的克隆代码分析与一致性维护方法,重点研究了克隆代码演化特征分析、克隆代码创建和变化的一致性需求预测和跨项目的克隆代码一致性预测,取得了如下创新性成果:

(1) 针对现有的克隆分析方法不能全面且客观的理解克隆代码及其演化过程,提出了一个基于-means聚类的克隆代码演化特征分析方法。通过检测软件版本中的克隆代码,并构建系统克隆家系描述克隆代码的演化过程。从克隆片段、克隆组和克隆家系三个不同的角度,提取相应的属性值表示克隆代码及其演化情况。在此基础上使用X-means聚类方法分析和挖掘克隆代码演化特征。实验结果表明软件中存在的大部分克隆代码在其演化过程中是较为稳定的,并不会发生频繁的变化。同时,也存在一定规模的发生变化的克隆代码,其中发生一致性变化的克隆代码要多于发生不一致变化的克隆代码。

(2)针对新创建的克隆代码在其演化过程可能会引发一致性变化,从而导致额外的维护代价问题,提出了克隆代码创建一致性维护需求预测方法。通过定义克隆代码创建一致性维护需求,将问题转化为可以应用机器学习方法解决的分类问题。通过构建系统克隆家系并识别克隆创建一致性变化模式,从系统中收集克隆创建实例用于训练机器学习模型,提取代码属性表示被复制的克隆代码、上下文属性表示被粘贴的克隆代码。在四个软件系统上进行了实验评估,实验结果表明该方法以较高的准确率和召回率有效地预测克隆代码创建一致性维护需求。

(3)针对演化中的克隆代码的一致性变化可能会导致一致性缺陷问题,提出了克隆代码变化一致性维护需求预测方法。通过克隆代码变化的一致性维护需求,将问题装化为可以应用机器学习方法解决的分类问题。通过构建系统的克隆家系和识别其全部的演化模式(尤其是克隆变化的一致性变化模式),从系统中收集克隆变化实例用于训练机器学习模型。从克隆组的角度提取了代码属性、上下文属性,并从克隆演化的角度提取了演化属性和变化信息共同表示克隆代码的变化。在四个软件系统上进行了实验评估,实验结果表明该方法以合理的准确率和召回率有效地预测克隆代码变化的一致性维护需求。

(4)针对软件开发初期系统训练数据不足的问题,提出了跨项目克隆代码一致性维护需求预测方法,并统一克隆创建、变化的一致性维护需求为克隆一致性维护需求。将软件系统划分为训练系统和测试系统,并使用训练系统训练机器学习模型,在测试系统上进行跨项目一致性维护需求预测。在四个软件系统上进行了实证研究,实验结果表明可以在软件开发初期对系统进行跨项目的克隆一致性维护需求预测,并根据实验结果给出了一些建议帮助提高跨项目的预测能力。同时,结合软件开发过程,设计并实现了一个基于eclipse的克隆一致性维护需求预测插件,可以在实际开发过程中,帮助程序开发人员边开发、边维护克隆代码,从而提高软件质量和可维护性。

在本文工作的基础上,还有以下工作有待于进一步研究:

(1) 在克隆变化一致性维护需求预测中,所构建模型的预测能力仍然有一定的改进空间。在未来研究中,拟通过改进现有的属性用于表示克隆代码的真实变化情况,进一步提高模型的预测能力,以更高的准确率和召回率预测克隆代码一致性维护需求。

(2)在跨项目的克隆一致性维护需求预测中,所提出方法的预测能力尚未达到令人满意的效果。在未来的研究中,拟通过提取与项目自身相关的属性、筛选与测试系统相似的数据,进一步提高模型的跨项目预测能力,从而在软件开发初期帮助缺乏训练数据的新系统进行克隆一致性维护需求预测。

(3)在软件开发过程中,克隆代码变化时,可以使用本文方法判断该变化是否会引发一致性变化,进而提醒程序开发人员对克隆代码进行一致性维护。但是,并没有帮助程序人员维护克隆代码的一致性。在未来的研究中,拟结合程序分析技术,在判别需要一致性维护的基础上,对克隆代码进行自动地一致性维护,从而避免相关缺陷和提高软件的可维护性。同时,还可以使用该方法识别和修复软件中存在的一致性缺陷,帮助提高软件质量。